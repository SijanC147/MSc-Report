% !TEX root = ../../fyp.tex
\documentclass[../../fyp.tex]{subfiles}

\begin{document}
A task represents the start of a new experiment or a continuation thereof and is invoked using the \texttt{tsaplay.task} submodule: 
\begin{code}
	python -m tsaplay.task single \ 
		--batch-size=25 \ 
		--steps=1000 \  
		--model="lstm" \  
		--embedding="wiki-50[corpus,only_adjectives]" \ 
		--datasets 'dong[33/34/33,15/70/15]' wang \ 
		--contd-tag="example-experiment-task" \ 
		--model-params hidden_units=50 oov=true num_oov_buckets=100 \ 
		--aux-config logging=false debug=cli \ 
		--run-config save_checkpoints_steps=1000 \
		--comet-api="XXXXXXXXXXXXXXXX" \
		--verbosity="INFO"
\end{code}

This section covers the CLI arguments available on this command when executing a single task.
\begin{itemize}
\item \texttt{--steps} The number of steps to train for.
\item \texttt{--batch-size} The number of samples to use in each batch.
\item \texttt{--model} The name of the model being run. 
%(might need an appendix explaining how to integrate a model, and why it had to be this way, because comet.ml needs to be imported before everythin else) 
\item \texttt{--embedding} The name of the embedding to use for the experiment. A comma-delimited list of filters may be included, as well as the \texttt{corpus} keyword to filter on tokens occurring in the datasets. 
\item \texttt{--datasets} One or more datasets specified by the name used to import them. Each dataset argument may be postfixed with a comma delimited list describing of redistribution percentages for training and testing. Each redistribution must have the same number of elements as the amount of unique classes in the dataset, separated using a \texttt{/} character. If only one set of values are provided this is applied to both training and testing datasets. 
\item \texttt{--model-parameters} Model hyperparameters which may override the default specified in its parameters function. The OOV policy parameters are also supplied through this parameter set.  
\item \texttt{--run-config} Values which are passed on to the Tensorflow run configuration. These include options such as setting a random seed for an experiment and how often to save model summaries during training.\footnote{https://www.tensorflow.org/versions/r1.12/api\_docs/python/tf/estimator/RunConfig}
\item \texttt{--aux-params} Auxiliary model parameters, separate from its hyperparameters, include programmatic switches for add-ons and a flag for enabling debug mode.
\item \texttt{--contd-tag} An optional name to index the experiment under. If the name already exists, that experiment will continue training up to the provided \texttt{--steps} argument.
\item \texttt{--comet-api} Comet.ml API key which enables real-time remote logging of the experiment (requires a \texttt{contd-tag} argument to be set). 
\item \texttt{--verbosity} Controls the logging level on \texttt{stdout} during execution.  
%\item job-dir A google cloud storage location to store the experiment when running on the google cloud service.
\end{itemize}
\clearpage
A batch option was also developed to allow multiple tasks to be run in sequence on the google cloud platform without the need to re-build and re-submit a job for each task. This command is also invoked on the same \texttt{tsaplay.task} submodule: 
\begin{code}
	python -m tsaplay.task batch "batch_tasks.txt"
\end{code}

The batch-file syntax allows the user to specify multiple tasks using the single task CLI format where each task is delimited by a new-line character. Additionally, comments may be included in the file using either the \enquote*{\texttt{\#}} or \enquote*{\texttt{;}} characters at the start of a new line.\footnote{An example batch file is included in the source code for reference.}

\subsection{Google Cloud Support}
Making use of a cloud computing service provides additional computational power through hardware such as GPUs which reduce experimentation times through increased parallel computing capabilities. Additionally, by running experiments in a remote environment, local machines can be used to continue development on future experiments. The Google cloud platform was chosen as it integrates well with the Tensorflow library, itself being a product of the same company. Moreover, this platform also provided us complimentary credits with which we could carry out our work. 

While the framework facilitates integration with the Google cloud AI platform, some pre-requisites must be met on the host machine. These include setting up Google cloud CLI tools, as well as a Google cloud project and cloud storage bucket. The specifics of this process are beyond the scope of this work and can be reviewed in the relevant Google cloud API documentation.\footnote{https://cloud.google.com/ml-engine/docs/}

Since the framework must be packaged as a module before submitting a job, an external script must be used to programmatically invoke this process: 

\begin{code}
	python submit_job.py \ 
		--job-id="example_job" \ 
		--job-dir="experiments_data" \
		--job-labels type=example \
		--machine-types masterType=standard_gpu workerType=large_model workerCount=1 \
		--stream-logs \
		--show-sdist \
		--task-args batch "batch_tasks.txt"
\end{code}

\begin{itemize}
\item \texttt{--job-id} A unique string which identifies the job being submitted. This is also used as a job-dir if one is not provided.
\item \texttt{--job-dir} The name of the parent directory which will house experiment data using the same structure described in \S\ref{sec:experiment_module} 
\item \texttt{--job-labels} Optional labels which may be attached to a specific job for organizational purposes.
\item \texttt{--machine-types} Used to specify the specific machine configuration for the job, as offered by the google cloud platform.\footnote{https://cloud.google.com/ml-engine/docs/machine-types} 
\item \texttt{--stream-logs} A flag which, when set, continues to stream log data to the stdout as the job runs.
\item \texttt{--show-sdist} A flag which, when set, logs the build process that takes place before submitting a job to the stdout.
\item \texttt{--task-args} The arguments that are forwarded to the \texttt{tsaplay.task} submodule which follows the same syntax described in \S\ref{sec:executing_tasks}.
\end{itemize}

The \texttt{submit\_job.py} script is responsible for packaging the framework with the requisite assets then submitting the job request to the Google cloud platform. This process must be repeated for every job that is submitted. Tasks that fall within the scope of the data module are carried out on the host machine, before packaging the framework and requisite assets. In this way, the framework can minimize the size of the assets that need to be uploaded to the cloud while storing this data for future reuse. The internal assets directory is used to store the requisite embeddings, datasets, and features for a particular job. It is packaged with the rest of the \texttt{tsaplay} directory as a distributable python module and uploaded to the Google cloud platform. This module is subsequently installed on the platform to invoke the \texttt{tsaplay.task} submodule.

At the time of writing, task execution across multiple machines in a distributed environment is considered experimental as some features, such as comet.ml logging, are known to malfunction in this setting. This is likely a consequence of machine instances submitting data to the comet.ml service concurrently. Attempts to rectify this were made, however the issue was ultimately deferred as it was beyond the scope of this work.
\end{document}