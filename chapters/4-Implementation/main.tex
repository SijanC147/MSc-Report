% !TEX root = ../../fyp.tex
\documentclass[../../fyp.tex]{subfiles}

\begin{document} 
This chapter presents a closer look at the implementation details of the tool-set the framework provides, as well as the interface used to interact with these tools and carry out experiments. Crucially, while the development of these tools is motivated by our own requirements in carrying out experiments with the framework, it is pursued with the overarching goal of providing an encompassing framework architecture that streamlines their use over a series of experiments, using a simplified user interface. 

\section{Importing Datasets} \label{sec:importing_datasets}
\subfile{chapters/4-Implementation/4.1-ImportingDatasets}

\section{Dataset Re-Distribution} \label{sec:dataset_redist}
\subfile{chapters/4-Implementation/4.2-DatasetRedistribution}

\section{Filtering Embeddings} \label{sec:filtering_embeddings}
\subfile{chapters/4-Implementation/4.3-FilteringEmbeddings}

\section{Datasets}
\subfile{chapters/4-Implementation/4.4-Datasets}

\section{Executing Tasks} \label{sec:executing_tasks}
\subfile{chapters/4-Implementation/4.5-ExecutingTasks}

The tools outlined herein provide functionality which may be programmatically invoked from the same interface used to carry out experiments or submit jobs. These implementation approach adheres to our objective of providing a simplified interface, minimizing the amount of work required for successive experimentation, while still producing substantive downstream outputs. In chapter \S\ref{chap:evaluation_methods} we provide an overview these outputs, from metrics and visualizations to integrations with useful analysis tools, which is followed by the results and observations from our OOV and reproducibility experiments in chapter \S\ref{chap:results_observations}.

\end{document}