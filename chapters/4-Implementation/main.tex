% !TEX root = ../../fyp.tex
\documentclass[../../fyp.tex]{subfiles}

\begin{document} 
This chapter presents a closer look at the implementation details of the tool-set the framework provides, as well as the interface used to interact with these tools and carry out experiments. Crucially, while the development of these tools is motivated by our own requirements in carrying out experiments with the framework, it is pursued with the overarching goal of providing an encompassing framework architecture that streamlines their use over a series of experiments, using a simplified user interface. 

\section{Importing Datasets} \label{sec:importing_datasets}
\subfile{chapters/4-Implementation/4.1-ImportingDatasets}

\section{Dataset Re-Distribution} \label{sec:dataset_redist}
\subfile{chapters/4-Implementation/4.2-DatasetRedistribution}

\section{Filtering Embeddings} \label{sec:filtering_embeddings}
\subfile{chapters/4-Implementation/4.3-FilteringEmbeddings}

\section{Datasets} \label{sec:datasets}
\subfile{chapters/4-Implementation/4.4-Datasets}

\section{Executing Tasks} \label{sec:executing_tasks}
\subfile{chapters/4-Implementation/4.5-ExecutingTasks}

\section{Outputs}
\subfile{chapters/4-Implementation/4.6-Outputs}

\section{Comet-ML Integration}
\subfile{chapters/4-Implementation/4.7-CometMLIntegration}

\section{Summary}
We have explored the details of each stage of the framework, from data preparation to model implementation to the output data and tools used to evaluate them. At each of these stages the core principles of the framework have been to enable rapid experimentation through an intuitive interface and extending this ease-of-use into an API which allows users to expand the functionality of the framework at various points of the process such as with novel datasets, embedding filtering techniques and, custom visualizations. This implementation approach adheres to our objective of providing a simplified interface, minimizing the amount of work required for successive experimentation, while still producing substantive downstream outputs. 

In the next chapter we present the results obtained from two separate use-cases of the framework, a reproducibility study for a number of models, implemented using the framework API and, an investigation into a number of OOV approaches using the same models.  

% In chapter \ref{chap:evaluation_methods} we provide an overview these outputs, from metrics and visualizations to integrations with useful analysis tools, which is followed by the results and observations from our OOV and reproducibility experiments in chapter \ref{chap:results_observations}.

\end{document}