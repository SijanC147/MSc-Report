% !TEX root = ../../fyp.tex
\documentclass[../../fyp.tex]{subfiles}

\begin{document}
Following the principles outlined by Moore \cite{moore2018}, and motivated by the observations on these principles made by Dhingra \cite{bhuwandhingra2017} in the field of reading comprehension and the effect of out-of-vocabulary(OOV) embedding strategies thereof, the objectives of this study are two-fold.

\begin{enumerate}
	\item We propose to extend the work carried out by Moore et al. \cite{moore2018} in the field of TSA to cover a wider range of studies, including those that employ techniques that have since emerged focusing on attention mechanisms and memory networks. This entails the attempt to reproduce these models based on the detail provided in the original studies and subsequently carrying out a comparative evaluation of these approaches across a wide range of datasets from varying domains using a number of different pre-trained word embeddings.
	\item Inspired by the work and findings of Dhingra et al. \cite{bhuwandhingra2017}, we shall endeavor to investigate the effect that different OOV embedding strategies and pre-trained word embeddings have on the downstream performance of models with respect to TSA. To our knowledge, at the time of writing, this study will be the first to investigate this issue in detail.
\end{enumerate}

To achieve these goals, we make three principal contributions through this work:
\begin{itemize}
	\item A publicly accessible framework that provides access to a range of frequently cited datasets that have been used in the field of TSA. This framework shall be used to obtain robust performance metrics, such as macro-f1 scores for all models, which have been hitherto unreported for a subset of the models, as well as other informative measures that shed light into the inner workings of the models implemented, such as attention heat-maps (where applicable).
	\item A comparative evaluation of models across different domains and datasets, using different pre-trained word embeddings to ascertain the degree to which results obtained are reproducible and generalizable.
	\item Detailed reports on a series of experiments using different OOV embedding strategies across all implemented models, the results of which will allow us to deduce the degree to which these affect downstream performance and whether an optimal approach can be found that proves to be generally beneficial.
\end{itemize}

Finally, the proposed framework shall also serve as the groundwork for future experimentation into alternative, more sophisticated, OOV embedding approaches while also providing a means of rapidly carrying out comparative evaluations of TSA models across different datasets and pre-trained word embeddings.


\end{document}