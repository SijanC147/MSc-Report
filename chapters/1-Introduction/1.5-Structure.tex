% !TEX root = ../../fyp.tex
\documentclass[../../fyp.tex]{subfiles}

\begin{document}
Prior to presenting the design and implementation details of our framework, we provide a more detailed description of the field of TSA in chapter \ref{chap:background}. First, we provide a wider view of the field in its current state, the challenges it must overcome as well as the metrics commonly used to measure its progress. We follow this with an account of the approaches employed to this task and the emergent concepts thereof which propelled the field forward, from manual feature engineering to sophisticated attention mechanisms. 

Chapter \ref{chap:design} illustrates the principles we adopt for the design of each constituent unit of our framework from a high-level point of view, how these units interact and, our motivations for these design choices. A more in-depth review follows in chapter \ref{chap:implementation} which details specifics of the implementation both of the framework as a whole and its more complex components. We wrap-up the framework specification in chapter \ref{chap:evaluation_methods} by listing the different analysis and evaluation outputs of the framework. 

We use our framework to carry out a series of experiments, covering both reproducibility and OOV handling, the results of which we include in chapter \ref{chap:results_observations}, followed by an analysis of these results and our observations. Finally, we present our concluding remarks in chapter \ref{chap:conclusion} accompanied by our views on possible avenues to pursue in the future development of this framework.


\end{document}