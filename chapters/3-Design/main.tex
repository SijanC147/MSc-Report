% !TEX root = ../../fyp.tex
\documentclass[../../fyp.tex]{subfiles}

\begin{document}
% A detailed explanation of how the problem was tackled along with justifications for all decisions taken in the course of the solution.
% Design chapter Intro
%     describe the conceptual design of the framework  
%     provides reasons for notable design choices made
%     Show image of the block diagram
%     Next sections cover design details of the Data and Experiment module, and their constituent components
%     Specific implementation details are given in chapter 4 
%     Chapter 5 describes the evaluation measures produced by the framework
\section{Data Module}
\subfile{chapters/3-Design/3.1-DataModule}

\section{Experiment Module} \label{sec:experiment_module}
\subfile{chapters/3-Design/3.2-ExperimentModule}

We have discussed the design principles that were considered at each sub-component on a conceptual level, to address the objectives described in \S\ref{sec:objectives}. Furthermore, we have presented how each of these sub-components assemble to establish a framework which simplifies the process of implementing new TSA models and provides a static evaluation environment with robust metrics and informative visualizations, illustrated in chapter \ref{chap:evaluation_methods}. First, however, chapter \ref{chap:implementation} shall expound on how the design principles outlined herein were actualized into code at particular stages of the framework.
\end{document}