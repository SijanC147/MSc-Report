% !TEX root = ../../fyp.tex
\documentclass[../../fyp.tex]{subfiles}

\begin{document}
Tensorboard is a development tool, included with the Tensorflow library, used primarily for visualizing the transient performance and behavior of a model at specific checkpoints during training and testing. A summary is a representation of a model at a particular checkpoint, encapsulating all the information about that model at that step in training. At each checkpoint, the model is run against the test dataset consisting exclusively of unseen samples and the respective evaluation metrics are added to the summary object of that checkpoint. The add-on mechanism wraps around these hooks to allow custom metrics and visualizations to be added to the summary object.

 The checkpoint frequency of an experiment is a tunable run configuration parameter. While more frequent checkpoints increase the granularity of the transient data for a model, this also incurs a cost on experimentation times, particularly when dealing with visualizations which may need to be generated multiple times. In certain situations however, such as employing early stopping, more frequent checkpoints may be desired to minimize the risk for overfitting.

The Tensorboard service is started using a CLI command pointing to the experiment directory, stored locally or on a google cloud storage bucket, and is subsequently made accessible through a local webserver.   

\subsection{Scalar Metrics}
Model loss is plotted during training and evaluation to identify possible cases of overfitting. Although not an ideal metric, accuracy is recorded to compare results with papers that report it exclusively. Mean per-class accuracy is also monitored as it is more robust to class imbalances, however, unlike the macro-F1, it does take into consideration false positives. Other metrics are recorded only for intermediary data that are produced in their calculation. The mean intersection-over-union (IOU), for instance, although not typically cited in TSA applications, produces a confusion matrix tensor internally.
\end{document}