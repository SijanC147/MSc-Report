\documentclass[12pt, a4paper]{report}

\usepackage{fyp}

%%these packages are not really necessary if you dont need the code and proofs environments
%%so if you like you can delete from here till the next comment
%%note that there are some examples below which obviously won't work once you remove this part
\usepackage{verbatim}
\usepackage{amsfonts}
\usepackage{amsmath}
\usepackage{amssymb}
\usepackage{amsthm}
\usepackage{acronym}
\usepackage{csquotes}
\usepackage{multirow}
\usepackage[ruled,vlined,linesnumbered,noresetcount]{algorithm2e}
\usepackage[table,xcdraw]{xcolor}
\usepackage[normalem]{ulem}
\useunder{\uline}{\ul}{}

\newcommand{\unk}{\textit{\textless{UNK}\textgreater} }
%%this environment is useful if you have code snippets
\newenvironment{code}
{\footnotesize\verbatim}{\endverbatim\normalfont}

%%the following environments are useful to present proofs in your thesis
\theoremstyle{definition}
\newtheorem{definition}{Definition}[section]
\theoremstyle{definition}%plain}
\newtheorem{example}{Example}[section]
\theoremstyle{definition}%remark}
\newtheorem{proposition}{Proposition}[section]
\theoremstyle{definition}%remark}
\newtheorem{lemma}{Lemma}[section]
\theoremstyle{definition}%remark}
\newtheorem{corollary}{Corollary}[section]
\theoremstyle{definition}%remark}
\newtheorem{theorem}{Theorem}[section]
%%you can delete till here if you dont need the code and proofs environments

\graphicspath{ {figures/}{../figures/} }

\setlength{\headheight}{15pt}
%\overfullrule=15pt

%% TEMPORARY FIX FOR MORE READABLE CITATIONS 
% \usepackage{natbib}
% \bibpunct{(}{)}{,}{a}{}{;}
% \bibliographystyle{abbrvnat}
% \renewcommand{\cite}[1]{[\citealp{#1}]}
%% END OF TEMPORARY FIX FOR MORE READABLE CITATIONS 
\bibliographystyle{abbrv} %% UNCOMMENT 
\makeatletter
\newcommand{\algorithmfootnote}[2][\footnotesize]{%
  \let\old@algocf@finish\@algocf@finish% Store algorithm finish macro
  \def\@algocf@finish{\old@algocf@finish% Update finish macro to insert "footnote"
    \leavevmode\rlap{\begin{minipage}{\linewidth}
    #1#2
    \end{minipage}}%
  }%
}
\makeatother
\usepackage{subfiles}
\begin{document}


%%make sure to enter this information
\title{TSAPLAY: A framework to aid reproducibility in and exploration of the field of Targeted Sentiment Analysis}
\author{Sean Bugeja}
\date{enter a date}
\supervisor{Mr. Mike Rosner}
\department{Faculty of ICT}
\universitycrestpath{crest}
\submitdate{enter a date}

\frontmatter


\begin{acknowledgements}
	your acknowledgments
\end{acknowledgements}

\begin{abstract}
	Provides a short (typically 1 page) overview of the dissertation's contents including the tackled problem and high-level results/conclusions.
\end{abstract}

\tableofcontents

\listoffigures

\listoftables

\newpage
\section*{Acronyms}
\begin{acronym}
	\acro{NLP}{Natural Language Processing}
	\acro{LSTM}{Long Short Term Memory}
	\acro{OOV}{Out of Vocabulary}
	\acro{IV}{In Vocabulary}
	\acro{SVM}{Support Vector Machine}
	\acro{ASR}{Automatic Speech Recognition}
	\acro{RC}{Reading Comprehension}
	\acro{LDA}{Latent Dirichlet Allocation}
	\acro{LSA}{Latent Semantic Analysis}
	\acro{CLI}{Command Line Interface}
\end{acronym}


\mainmatter

%%you can organize your chapters into parts but this is not always necessary
%%\part{Part1} - Available but generally not used
\chapter{Introduction} \label{chap:introduction}
\subfile{chapters/1-Introduction/main}

\chapter{Background and Literature Review} \label{chap:background}
\subfile{chapters/2-LiteratureReview/main}

\chapter{Design} \label{chap:design}
\subfile{chapters/3-Design/main}

\chapter{Implementation} \label{chap:implementation}
\subfile{chapters/4-Implementation/main}

\chapter{Evaluation Methods} \label{chap:evaluation_methods}
\subfile{chapters/5-EvaluationMethods/main}

\chapter{Results and Observations} \label{chap:results_observations}
\subfile{chapters/6-Results/main}

\chapter{Conclusion} \label{chap:conclusion}
\subfile{chapters/7-Conclusion/main}

\appendix

% \begin{proof}
% this is a proof
% \end{proof}

% \chapter{Appendix A}
% \section{These are some details}
% %%example of the code environment
% \begin{code}
% 	this is some code;
% 	Make sure to use this template.
% \end{code}


\bibliomatter



\bibliography{references}
\end{document}
